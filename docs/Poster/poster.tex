\documentclass{beamer}
\usepackage{tikz}
\usepackage[orientation=landscape,size=a2,scale=1]{beamerposter}
\usepackage[absolute,overlay]{textpos}
\usetikzlibrary{arrows}
\begin{document}

\begin{textblock}{15}(0.5, 0.5)
    \begin{block}{}
        \centering
        \Large Suppression of Variation in Cell-Size: A Control Theoretic Approach \\
        \large Dilawar Singh, \texttt{dilawars@ncbs.res.in}
    \end{block}
    \begin{block}{Abstract}

        We propose a  possible mechanism based on control and information theory
        which can be used to control cell size. We explore few network
        topologies of a simple control network which can keep the size of the
        cell at a fixed value while giving a upper bound on the size of the
        Endosome.

\end{block}
\end{textblock}

\begin{textblock}{7}(0.5,3.5)

    \begin{block}{Network under study}

        \begin{figure}
        \begin{center}
        \begin{tikzpicture}[scale=1]

            %% Here is the macro which draws the gene regulatory arrow.
            %% args: starting point, width, height, fillcolor
            \newcommand\arrow[4]
            {
                \draw[color=#4,fill] #1  -- ++(0,0.5) -- ++(-5,0);
            };

            \begin{scope}
                \arrow{(0,0)}{}{}{red};
            \end{scope}

            
        \end{tikzpicture}
        \end{center}
        \caption{}
        \label{fig:}
        \end{figure}

        
    \end{block}

\end{textblock}

\end{document}
